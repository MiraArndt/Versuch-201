\section{Durchführung}
\label{sec:Durchführung}
Um die spezifischen Wärmekapazitäten zu messen wird
ein Dewargefäß verwendet. Für eine möglichst exakte
Auswertung wird also zunächst $c_g m_g$ des
Dewargefäßes bestimmt. Dazu werden zunächt ca 300 ml 
Wasser in das Gefäß gefüllt und anschließend die
Temperatur gemessen. Weitere ca 300 ml
werden erhizt und ins Gefäß gegeben.
Nachdem sich eine konstante Temperatur eingestellt hat wird diese gemessen.
Aus den gemessenen Größen der Mssen und Temperaturen lässt sich nun
$c_g m_g$ berechnen.\\


\noindent Da die Wärmekapazität des Dewargefäßes von seiner Füllhöhe abhängt
werden bei den folgenden Messungen auch jeweils ca 600 ml verwendet.
Es werden Kupfer, Aluminium (mit jeweils 3 Messungen)
und Graphit (Mit einer Messung) untersucht.
Zunächst werden die Proben gewogen und anschließend auf
einer Heizplatte in einem Wasserbad erhizt. 
Gleichzeitig werden jeweils ca 600 ml Wasser in
das Dewargefäß gegeben, dessen Masse bestimmt und die Temperatur im Gefäß gemessen.
Nun wird die Probe mit in das Dewargefäß gegeben 
und leicht umgerührt. Nachdem sich eine konstante Temperatur eingestellt hat wird auch diese gemessen.
