\section{Diskussion}
\label{sec:Diskussion}

Es fällt auf, dass die errechneten Werte für die spezifischen Wärmekapazitäten der Proben 
Aluminium und Graphit nur geringfügig von ihren jeweiligen Literaturwerten abweichen. Der 
errechnete Wert für die spezifische Wärmekapazitäten von Kupfer weicht mit $10,91\,\%$ etwas
weiter von ihrem Literaturwert ab, ist jedoch trotzdem akzeptabel. Die Abweichungen aller Werte 
lassen sich wohlmöglich darauf zurückführen, dass die Genauigkeit der aufgenommenen Messwerte in 
Frage zu stellen ist. Beispielweise wurde vernachlässigt, dass der Körper bereits beim Übersetzen
vom erhitzten Wasser ins Dewargefäß ein wenig Wärme verloren hat oder auch, dass die Probe sobald 
sie sich im Dewargefäß befindet auch ein wenig Wärme an an ihre Umgebung abgegeben hat. Die 
errechneten Werte für die Molwärmen mithilfe der Kupfer- und Aluminiumprobe liegen sehr nah an 
dem vom Dulong-Petitschen Gesetz beschriebenen Wert und scheinen somit die Gültigkeit der 
klassischen Beschreibung zu bestätigen. Im Gegensatz dazu weicht der errechnete Wert für die 
Molwärme mithilfe der Graphitprobe um knapp $67\, \%$ vom dem Literaturwert ab. Diese Abweichung
kann nicht auf Messungenauigkeiten zurückgeführt werden, da der errechnete Wert für die 
spezifische Wärmekapazität von Graphit sehr nah am dem zugehörigen Theoriewert liegt.
Folglich kann der in diesem Teil bestimmte Wert nicht als Bestätigung der klassischen Beschreibung
angesehen werden, was bedeutet, dass an dieser Stelle eine quantenmechanische Betrachtung 
notwendig ist.





\begin{table}[h]
\normalsize

\centering
\sisetup{table-format=4.0}
\begin{tabular}{c c c c}
\toprule
        Probe & $c_{Probe} \,/\, \si{\joule\per\gram\per\kelvin}$ & $ c_{Literaturwert} $\cite{stocker2004taschenbuch}$\,/\, \si{\joule\per\gram\per\kelvin}$ & Prozentualer Fehler$\,/\, \% $\\
        \midrule
        Kupfer      &   0,427   &   0,385   &   10,91\\
        Aluminium   &   0,917   &   0,897   &    2,23\\
        Graphit     &   0,690   &   0,710   &   -2,82\\

\bottomrule

\end{tabular}

\caption{Aufführung der Werte für die spezifische Wärmekapazitäten}
\label{tab:werte1}
\end{table}



\begin{table}[h]
\normalsize

\centering
\sisetup{table-format=4.0}
\begin{tabular}{c c c c}
\toprule
        Probe & $C_{V,Probe} \,/\, \si{\joule\per\mole\per\kelvin}$ & $ C_{V,Literaturwert}$ \cite{stocker2004taschenbuch}$,/\, \si{\joule\per\mole\per\kelvin}$ & Prozentualer Fehler$\,/\, \% $\\
        \midrule
        Kupfer      &   26,36   &   24,9   &     5,86\\
        Aluminium   &   23,64   &   24,9   &    -5,06\\
        Graphit     &   8,250   &   24,9   &   -66,86\\

\bottomrule

\end{tabular}

\caption{Aufführung der Werte für die gemessene und errechnete Molwärme}
\label{tab:werte2}
\end{table}