\section{Theorie}
\label{sec:Theorie}

\subsection{spezifische Wärmekapazität und Atomwärme}
Unter der Atomwärme $C$ eines bestimmten Stoffes 
versteht man eine bestimmte
Wärmemenge $\symup{d}Q$, die erforderlich ist, um
ein Mol des Stoffes um $\symup{d}T$ zu erwärmen.
Da die Wärmemenge eine Form der Energie darstellt,
gilt $\symup{d}Q = \symup{d}U$, wobei $U$ der
inneren Energie eines Mols des Stoffes entspricht.


\noindent Bezieht sich nun diese Größe auf die Masse des Stoffes,
so ergibt sich die spezifische Wärmekapazität $c$ mit

\begin{equation}
    \Delta Q=mc\Delta T.
\end{equation}


\noindent Da beide Größen von den äußeren Bedingungen der
Wärmezufuhr abhängen wird zwischen der Atomwärme
und spezifischen Wärmekapazität bei gleichem Volumen

\begin{equation}
    C_V=\frac{\symup{d}Q}{\symup{d}T}\biggr|_V=\frac{\symup{d}U}{\symup{d}T}\biggr|_V
    \label{eq:b}
\end{equation}

\begin{equation}
    c_V=\frac{\symup{d}Q}{m \cdot \symup{d}T}\biggr|_V=\frac{\symup{d}U}{m \cdot \symup{d}T}\biggr|_V
\end{equation}
\noindent und gleichem Druck

\begin{equation}
    C_P=\frac{\symup{d}Q}{\symup{d}T}\biggr|_P=\frac{\symup{d}U}{\symup{d}T}\biggr|_P
\end{equation}

\begin{equation}
    c_P=\frac{\symup{d}Q}{m \cdot \symup{d}T}\biggr|_P=\frac{\symup{d}U}{m \cdot \symup{d}T}\biggr|_P
\end{equation}

\noindent unterschieden. Zwischen $C_V$ und $C_P$
besteht außerdem der Zusammenhang

\begin{equation}
    C_P-C_V=9\alpha^2\kappa V_0T,
\end{equation}

wobei $\alpha$ einem linearen Ausgleichskoeffizient,
$\kappa$ dem Kompressionsmodul und $V_0$ dem Molvolumen entspricht.




\subsection{klassische Beschreibung}
Die über einen Zeitraum $\tau$ gemittelte innere 
Energie eines Atoms beträgt

\begin{equation}
    \bigl< u\bigr>=\frac{1}{\tau}\int_0^{\tau}u(t)\,\symup{d}t=\bigl< E_{kin}\bigr>+\bigl< E_{pot}\bigr>.
\end{equation}

\noindent befindet sich das Atom in einem Festkörper,
so kann es dort nur Schwinungen um eine Gleichgewichtslage 
ausführen, da es durch Gitterkräfte gebunden ist. Es
wird angenommen, dass es sich bei der Schwingung um eine
harmonische handelt, weshalb der Zusammenhang

\begin{equation}
    \bigl< E_{kin}\bigr>=\bigl< E_{pot}\bigr>
\end{equation}

\noindent gilt. Aus dem Äquipartitionstheorem folgt

\begin{equation}
    \bigl< E_{kin}\bigr>=\frac{1}{2}kT,
\end{equation}

\noindent pro Bewegungsfreiheitsgrad, wobei k der Boltzmannschen Konstante
entspricht und T die äußere Temperatur angibt. Da ein Atom im Festkörper in drei senkrecht aufeinander liegende
Richtungen schwingen kann gilt

\begin{equation}
    \bigl< u\bigr>=2\bigl< E_{kin}\bigr>=3kT.
    \label{eq:a}
\end{equation}

\noindent Wird nun statt eines Atoms ein Mol des Stoffes
betrachtet so ändert sich \ref{eq:a} zu

\begin{equation}
    \bigl< U\bigr>=2\bigl< E_{kin}\bigr>=3RT
\end{equation}

\noindent mit R als allgemeine Gaskonstante. Aus Gleichung \ref{eq:b}
folgt nun das Dulong-Petitsche Gesetz

\begin{equation}
    C_V=3R.
\end{equation}



\subsection{quantenmechanische Beschreibung}
Im Gegensatz zum Dulong-Petitschen Gesetz wurde beobachtet, dass die
Atomwärme eines Stoffes bei hinreichend tiefen Temperaturen beliebig klein werden.
Außerdem erreichen manche Stoffe mit geringem Atomgewicht
den Wert $C_V=3R$ erst bei sehr hohen Temperaturen von
teilweise über 1000 °C. Diese Effekte lassen sich durch eine
quantenmechanische Beschreibung erklären. Dabei wird davon ausgegangen,
dass die mittlere Energie $\bigl< E_{qu}\bigr>$ eines Atoms nicht mehr
linear von $T$ abhängt. Stattdessen gilt der Zusammenhang

\begin{equation}
    \bigl < u_{qu} \bigr > =\frac{\hbar \omega}{\exp(\frac{\hbar \omega }{kT})-1}
\end{equation}

\noindent für einen Freiheitsgrad. Für ein Mol des Stoffes
ergibt sich 

\begin{equation}
    \bigl < U_{qu} \bigr > =\frac{3N_L\hbar \omega}{\exp(\frac{\hbar \omega }{kT})-1},
\end{equation}

\noindent wobei $N_L$ der Anzahl von Atomen in einem Mol des Stoffes entspricht.
Für hohe Temperaturen nähert sich $\bigl < U_{qu} \bigr >$ dem klassischen Wert $3RT$ an,
für die meisten Stoffe ist dies bereits bei Zimmertemperatur erreicht.
